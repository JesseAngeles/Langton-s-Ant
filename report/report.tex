%Config
\documentclass[12pt,twoside]{article}
\usepackage[spanish,es-tabla]{babel}
\usepackage[a4paper]{geometry}

\usepackage{graphicx}               % Para incluir imágenes
\usepackage{amsmath}                % Para el manejo de matemáticas
\usepackage{url}
\usepackage{enumitem}

\title{La hormiga de Langton y la interacción entre diferentes hormigueros}
\author{Erick Jesse Angeles López}


% Definir un comando para palabras clave
\newcommand{\keywords}[1]{%
	\begin{center}
		\textbf{Palabras clave:} #1
	\end{center}
}

\renewcommand{\baselinestretch}{1}
\setcounter{page}{1}
\setlength{\textheight}{21.6cm}
\setlength{\textwidth}{14cm}
\setlength{\oddsidemargin}{1cm}
\setlength{\evensidemargin}{1cm}
\pagestyle{myheadings}
\thispagestyle{empty}
\markboth{\small{Ángeles López Erick Jesse}}{\small{La hormiga de Langton y la interacción entre diferentes hormigueros}}
\date{}

\begin{document}
	
	\begin{center}
		
		% Contenido izquierdo - Imagen
		\begin{minipage}{0.17\textwidth}
			\centering
			\includegraphics[width=0.7\textwidth]{img/ipn_logo.jpg} % Ajusta esta línea
		\end{minipage}
		\begin{minipage}{.55\textwidth}
			\centering
			{\Large Instituto Politécnico Nacional}\\
			{\Large Escuela Superior de Cómputo}
		\end{minipage}
		\begin{minipage}{0.17\textwidth}
			\centering
			\includegraphics[width=0.9\textwidth]{img/escom_logo} % Ajusta esta línea
		\end{minipage}			
	\end{center}
	
	
	\centerline{\bf Ingeniería en Inteligencia Artificial, Algoritmos bioinspirados}
	
	\centerline{\bf  Sem: 2025-1, 5BM1, Fecha: 6-Enero-2025}
	
	\centerline{}
	
	%\centerline{}
	
	
	\begin{center}
		\Large{\textsc{La Hormiga de Langton y la interacción entre diferentes hormigueros}} 
	\end{center}
	\centerline{}
	\centerline{\bf {\textit{Presenta}}}
	\centerline{\bf {Angeles López Erick Jesse\footnote{eangelesl1700@alumno.ipn.mx}}}
	\centerline{}
	\centerline{}
	\centerline{\bf {Disponible en:}}
	\centerline{\text{\url{GITHUB}}}
	
	
	
	
	\newtheorem{Theorem}{\quad Theorem}[section]
	
	\newtheorem{Definition}[Theorem]{\quad Definition}
	
	\newtheorem{Corollary}[Theorem]{\quad Corollary}
	
	\newtheorem{Lemma}[Theorem]{\quad Lemma}
	
	\newtheorem{Example}[Theorem]{\quad Example}
	
	\bigskip
	
	\bigskip
	
	\begin{abstract} 
		
	\end{abstract}
	
	\keywords{}
	
	\clearpage
	
	\tableofcontents
	\clearpage
	
	\clearpage
	\section{Introducción}
	
	La hormiga de Langton, creada por Chris Langton en 1986, es un modelo computacional sencillo que genera comportamientos emergentes y patrones complejos a partir de reglas simples. Originalmente, la hormiga sigue dos reglas básicas que dan lugar a un comportamiento caótico e impredecible. Este modelo es un excelente ejemplo de sistemas dinámicos en los que reglas locales producen estructuras globales.
	
	En esta práctica, exploramos el comportamiento de la hormiga de Langton bajo diferentes condiciones, como cambios en las reglas, la interacción de múltiples hormigas en un mismo espacio y el establecimiento de colonias. Nuestro objetivo es analizar las dinámicas emergentes y las densidades de los estados en el tablero, así como estudiar la expansión del área inicial ocupada por las hormigas. Para ello, se diseñó un software específico que permite realizar simulaciones personalizadas y graficar los resultados.
	
	\section{Marco Teórico}
	
	\subsection{Funcionamiento de la Hormiga de Langton}
	
	La hormiga de Langton se desplaza sobre una cuadrícula bidimensional (teóricamente infinita) en la que cada celda puede tener un estado de 0 o 1. Sigue dos reglas simples:  
	\begin{enumerate}
		\item Si la hormiga está en una celda con estado 0:
		\begin{itemize}
			\item Cambia el estado de la celda a 1.
			\item Gira 90° a la derecha y avanza.
		\end{itemize}
		\item Si la hormiga está en una celda con estado 1:
		\begin{itemize}
			\item Cambia el estado de la celda a 0.
			\item Gira 90° a la izquierda y avanza.
		\end{itemize}
	\end{enumerate}
	
	Esta configuración inicial se conoce como la regla \textbf{RL}, donde:  
	\begin{itemize}
		\item El tamaño de la regla define el número de estados consecutivos, en este caso $\{0, 1\}$.  
		\item \textbf{R} y \textbf{L} indican giros a la derecha (\textbf{Right}) e izquierda (\textbf{Left}), respectivamente.  
		\item El nuevo estado es el consecutivo en el ciclo. Por ejemplo, del estado 0 pasa al estado 1, y del estado 1 regresa al estado 0.  
	\end{itemize}
	
	Bajo esta estructura, pueden definirse múltiples reglas, como la regla \textbf{LLR}, que sigue estas instrucciones:  
	\begin{itemize}[noitemsep]
		\item Si el estado es 0, gira a la izquierda.  
		\item Si el estado es 1, gira a la izquierda.  
		\item Si el estado es 2, gira a la derecha.  
	\end{itemize}
	
	\subsection{Clasificación: Agente y Autómata Celular}
	
	La hormiga de Langton puede ser catalogada tanto como un agente como un autómata celular:  
	\begin{itemize}
		\item \textbf{Agente}: Es una entidad autónoma que toma decisiones individuales basadas en ciertas reglas, lo que coincide con el comportamiento definido anteriormente de la hormiga.  
		\item \textbf{Autómata celular}: Un sistema dinámico compuesto por celdas que cambian de estado según una vecindad y una función de transición. Aunque la definición no parece encajar perfectamente, el comportamiento de la hormiga puede replicarse definiendo estados y reglas específicas.  
	\end{itemize}
	
	Por ejemplo, para la regla \textbf{RL}, el autómata celular equivalente se define de la siguiente manera: 
	\begin{itemize}
		\item \textbf{Dimensión}: Plano bidimensional.  
		\item \textbf{Estados}: Se extienden los estados originales $\{0, 1\}$ para incluir información sobre la dirección de la hormiga: $\{0, 0_U, 0_D, 0_L, 0_R, 1, 1_U, 1_D, 1_L, 1_R\}$, donde el subíndice indica la dirección.  
		\item \textbf{Vecindad}: Vecindad de Neumann (los cuatro vecinos ortogonales más cercanos y la celda central).  
		\item \textbf{Función de transición}: Las reglas incluyen:  
		\begin{itemize}
			\item Si una celda contiene una hormiga, su estado cambia al estado sin hormiga:  
			\[ c(t) = x_y \rightarrow c(t+1) = x \]  
			\item Si una celda vecina contiene una hormiga mirando hacia ella, adquiere el estado correspondiente con la nueva dirección. Por ejemplo, a continuación se define el comportamiento de una celda si su vecina izquierda tiene una hormiga mirando a la derecha:  
			\[
			c_{(-1, 0)}(t) = x_R, c(t) = 0 \rightarrow c(t+1) = 1_D, \: \forall x \in \text{Estados iniciales}
			\]
			\[
			c_{(-1, 0)}(t) = x_R, c(t) = 1 \rightarrow c(t+1) = 0_U, \: \forall x \in \text{Estados iniciales}
			\]
		\end{itemize}
	\end{itemize}
	
	Este autómata celular esta diseñado unicamente para una hormiga.
	
	\subsection{Objetivo de la Práctica}
	
	En esta práctica, se busca estudiar:  
	\begin{itemize}
		\item El impacto de diferentes reglas sobre el comportamiento de las hormigas.  
		\item Las interacciones entre múltiples hormigas en un mismo espacio.  
		\item La formación, el comportamiento y la interacción entre colonias independientes.  
	\end{itemize}
	
	Durante las pruebas, se analizarán y graficarán las densidades de los estados de las celdas y la expansión desde el punto inicial. Las simulaciones se ejecutaron en un software diseñado específicamente para esta práctica, detallado en la sección de metodología.
	
	\section{Metodología}
	
	\subsection{Diseño de software}
	
	Se diseña un programa en c++ implementado el proyecto \textit{Graphic Mode} que permite diseñar interfaces graficas usando la la libreria SFML.
	
	\subsection{Comportamiento de una sola hormiga de Langton}
	
	\subsection{Interacción de multiples hormigas en el mismo espacio}
	
	\subsection{interacción entre multiples hormigueros}
	

	\clearpage
	\addcontentsline{toc}{section}{Referencias}
	\begin{thebibliography}{99}
		\bibitem{}
		
	\end{thebibliography}
	
\end{document}
